%%%%%%%%%%%%%%%%%%%%%%%%%%%%%%%%%%%%%%%%%%%%%%%%%%%%%%%%%%%%%%%%%%%%%%%%%%%%%%%%%%%%
% Do not alter this block (unless you're familiar with LaTeX
\documentclass{article}
\usepackage[margin=1in]{geometry} 
\usepackage{amsmath,amsthm,amssymb,amsfonts, fancyhdr, color, comment, graphicx, environ,minted}
\usepackage{xcolor}
\usepackage{mdframed}
\usepackage[shortlabels]{enumitem}
\usepackage{indentfirst}
\usepackage{hyperref}
\hypersetup{
    colorlinks=true,
    linkcolor=blue,
    filecolor=magenta,      
    urlcolor=blue,
}


\pagestyle{fancy}


\newenvironment{problem}[2][Problem]
    { \begin{mdframed}[backgroundcolor=gray!20] \textbf{#1 #2} \\}
    {  \end{mdframed}}

% Define solution environment
\newenvironment{solution}{\textbf{Solution}}

%%%%%%%%%%%%%%%%%%%%%%%%%%%%%%%%%%%%%%%%%%%%%
%Fill in the appropriate information below
\lhead{NAME - STUDENT ID NUMBER}
\rhead{STAT6181} 
\chead{\textbf{ASSIGNMENT \#1}}
%%%%%%%%%%%%%%%%%%%%%%%%%%%%%%%%%%%%%%%%%%%%%

\renewcommand{\labelenumiii}{(\roman{enumiii})}
\begin{document}
    
    \textbf{Due Date: 23rd September 2021}
    
    Instructions:
    
    \begin{enumerate}
        \item Answer ALL questions in the spaces allocated.
        \item In this assignment, you are required to show all your working. 
        \item Your answers must be written in the spaces provided. You can adjust the spaces allocated for the answers if you need more space. You can type your answers if you wish. 
        \item \emph{\underline{The lecturer maintains the right to call students in individually and ask them questions on the assignments.}} 
        
        \emph{\underline{This may result in an adjustment of the final assignment grade.}}
        
        \item \emph{\underline{Upload  (i) Your R code (ii) Your Data and (3) A softcopy of your assignment on myelearning as a pdf. }} 

        \emph{\underline{In Dropbox 1. DO NOT SUBMIT AS A SINGLE ZIP FILE with all the documents.}}
    \end{enumerate}
    
    \vspace{2cm}
    
    \begin{enumerate}
        \item QUESTION 1  (OPTIMIZATION – function with two variables)
            \begin{enumerate}
                \item Do some reading on how you can find the maximum and minimum of a function f(x,y). You may come across terms like saddles point, partial derivatives etc. After you have read that, find the maximum, minimum and saddle points (if any) exists for the following function:
        
                $$
                    f(x, y) = e^{-\frac{1}{3}x^2 + x + y^3}
                $$
                
                \hfill[10 marks]
                \item Use R to plot a the function $f(x,y)$. Use the following range $ -2 < x < 2$  and $-2 < y < 2$.
                    \begin{enumerate}
                        \item Rcode   
                            \hfill [2 marks]
                        \item Plot
                            \hfill [2 marks]
                    \end{enumerate}
                \item Use the optim function in R to confirm your results in (a).    
                    \begin{enumerate}
                        \item R code \& Output. 
                            \hfill [2 marks]
                    \end{enumerate}
            \end{enumerate}
        \item Find the two parameter Weibull distribution in Wikipedia.
            \begin{enumerate}
                \item Write some R code to plot this distribution for the parameters $b = 5$ and $\eta = 1$.
                \begin{enumerate}
                    \item R code 
                        \hfill[1 mark]  
                    \item Paste the plot here 
                        \hfill[1 mark]
                \item Use the newton’s method in the course handout and write suitable functions for $f, df$ and $df2$. Then generate 1,000 variates from a Weibull with $b = 5$ and $\eta = 1$ and check how well newton’s method is able to recover the estimates of the parameters.   You will follow the following steps to get this done:
                    \begin{enumerate}
                        \item code for generating the 1,000 variates from a Weibull with $b = 5$ and $\eta = 1$.   Use \texttt{set.seed(1123)} to ensure everyone has the same data.      
                        
                            \hfill [1 mark]
                        
                        \item Find the score vector (column of first derivatives).  
                        
                             \hfill [4 marks]
                        
                        \item Write an R function for the first derivatives $(df)$ which you will use in the Newton Raphson method.   
                            \hfill[ 4 marks]
                            
                        \item Find the hessian matrix (matrix of second derivatives which we call $df2$).         
                            \hfill[ 4 marks]
                        \item Write an R function for the hessian matrix $(df2)$ which you will use in the newton Raphson method.                                                                                                                   
                            \hfill[ 4 marks]
                        \item Apply the Newton Raphson Method in R and give the output. Comment on your results.   
                            \hfill[ 4 marks]
                    \end{enumerate}
                \end{enumerate}
            \end{enumerate}
    \end{enumerate}
    
    % \section*{\centering }
    % \begin{problem}{a}
    %     (a)	
        
    % \end{problem}
    % \vspace{1cm}
    % \begin{solution}
    
    % \end{solution}


\end{document}